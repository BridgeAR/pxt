\begin{abstract}

    With a drive to increase computer science literacy, embedded systems have entered the classroom. Historically, embedded systems in this setting require: knowledge of low-level programming languages; local installation of compilation toolchains, device drivers, and applications on PCs; and constant tethered connections to PCs in order for programs to execute. For students and educators, these requirements introduce barriers and restrictions in the classroom.

    We present a new programming platform that makes it easier for students and educators to create applications for embedded systems using higher level languages. We leverage today's lightweight tooling to allow code to be developed without \emph{any installation}, requiring only a computer with a modern web browser and a USB port. Finally, we describe architectural decisions that we have taken to facilitate long, free running Internet of Things applications, enabling new classroom scenarios for students and educators to explore.

    With the predicted growth of the Internet of Things we believe that a system such as ours, which enables students and educators from diverse backgrounds to develop embedded applications quickly, will become increasingly valuable. After describing our platform in detail, we evaluate performance on a range of modern low-end embedded systems, from 8-bit to 32-bit cores.

\end{abstract}
