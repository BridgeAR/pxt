
\subsection{Packages and Custom Editors}

Packages are \MCN's dynamic/static library mechanism for extending a target (by adding new code/data to the device
and simulator runtimes, as well as accompanying documentation). The following package extends the micro:bit target so
that the micro:bit can drive a NeoPixel strip of RGB LEDs: \dburl{https://github.com/Microsoft/pxt-neopixel}. 

To see how this
package is surfaced to the end-user, visit \dburl{http://makecode.microbit.org/} and select the ``Add Package'' option from the
gear menu; you will see the package ``neopixel'' listed in the available options. If you click on it, a new block category
named ``Neopixel'' will be added to the editor. In this scenario, PXT dynamically loads the (white listed) neopixel 
package directly from GitHub, compiles it and incorporates it into the web app. Packages also can be bundled with a web
app (the analog of static linking).  

For packages containing C++ files, a new C++ runtime has to be compiled in the cloud.
This is achieved by collecting all C++ files (in all packages plus the \CO runtime),
computing a hash, checking in the local in-browser cache if such a runtime was retrieved
before, and otherwise asking the cloud service to compile it.
Of course, the cloud service will return a cached copy, if the same set of C++
files was compiled before. The cache hit rates, in both the browser cache
and the cloud cache, are very high. 
The first hit rate is high because the cloud needs to be consulted
only when a new package is added to the project, and this particular combination 
of packages was never used before on current machine.
The second hit rate is high because people rarely combine many packages (due to
hardware and memory constraints), and there are only so many of them.