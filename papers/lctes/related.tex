\section{Related Work}
\label{sec:related}
%% potentially cut this
\subsection{Novice programming environments}

Arduino~\cite{buildingArduino2014} is an environment for programming microcontrollers, aimed at novices. However, its c-based APIs introduces barriers for novice programmers~\cite{blikstein2013gears}. Scratch~\cite{ScratchCACM2009} is a widely adopted, event-based visual programming environment environment designed to introduce novice programmers to computer science concepts. Extensions enable the programming of physical devices from Scratch. However, devices require constant tethered connections to operate, restricting potential projects~\cite{dougherty2012maker}. ArduBlock~\cite{Ardubloc28:online} brings visual programming to the Arduino, but it lacks the event-based blocks Scratch users are familiar with.

With the environments above, additional software must be installed, which creates barriers for novice users in restrictive environments. Our platform requires \emph{no installation} to support a diverse user base, and supports \emph{event-based higher level languages} to scaffold the progression of users to the world of the microcontroller.

\subsection{Virtual machine-based languages}

Recently, virtual machines supporting most of the semantics of higher level languages like JavaScript, Java, and Python, have been ported to 32-bit microcontrollers by maker~\cite{dougherty2012maker} communities. Examples include: micropython~\cite{MicroPython}, CircuitPython, and Espruino~\cite{espruinoBook}. These VMs consume a large amount of RAM and flash memory, and run significantly slower than native languages.

In the research community, the motive to bring higher-level languages to microcontrollers exists~\cite{koshy2005vmstar,st2009picobit,vaugon2015programming}. Rather than running a full-featured VM, others enable higher level languages to run efficiently by stripping out advanced language features, in favour of efficient, native execution~\cite{varma2004java}. Comparing these solutions to our solution is challenging due to a misalignment in evaluation metrics and microcontrollers. For example PICOBit uses an 8-bit MCU, and evaluates the cost of a VM, without the cost of a runtime environment. Simply accounting for a 32-bit MCU in this case, results in factor of 4 multiplication of most metrics.

Our approach bears most similarity to~\cite{varma2004java}, where we compile higher level languages to an \emph{optimised, event-driven} C++ runtime (\CON).

\subsection{Embedded runtime environments}

%Embedded runtime environments can range from simple C++ classes that control hardware, to real-time operating systems (RTOSs) with scheduling and memory management.

Arduino~\cite{buildingArduino2014} is an example of a simple platform where the developer uses high-level APIs to control hardware; there is no scheduler, and memory management is discouraged through a heavy emphasis on global variables.

TinyOS~\cite{levis2005tinyos}, Contiki~\cite{dunkels2012contiki}, RIOT OS~\cite{baccelli2013riot}, Mynewt~\cite{ApacheMy53:online} mbed OS~\cite{ARMmbed}, Zephyr~\cite{HomeZeph63:online} are RTOS solutions known widely in the systems community. The majority focus on the networking features of sensor based devices and commonly adopt a preemptive scheduling model, which leads to competition over resources resolved using locks and condition synchronization primitives. Contiki has a cooperative scheduler, but uses proto-threads to store thread context --- local variables are not allowed as the context of the stack is not stored.

Although the platforms above are widely used by C/C++ developers, none of these existing solutions align well with the programming paradigms seen in higher level languages. \CO \emph{bridges that semantic gap between the higher level language and the microcontroller}, offering appropriate abstractions and higher level primitives written natively in C++.

\subsection{Flashing microcontrollers}

There are two common ways to transfer a program to the flash of a microcontroller: for embedded developers, a specialized debugger chip; for hobbyists, a custom serial protocol~\cite{AVRDUDEA15:online}. Both approaches require operating system drivers. ARM's mbed platform provides DAPLink~\cite{GitHubAR5:online}, firmware that presents itself to an external computer as a USB pen drive. DAPLink exposes a virtual file system that caters for normal file system behaviour, and handles the decoding of Intel HEX files~\cite{IntelHEX} --- the firmware consumes 66 kB of flash and 13 kB RAM.

\UF contributes a new file format that \emph{greatly simplifies} the virtual file system approach, reducing complexity of the firmware, and thusly code size.



% talk about old protocols for flashing protocols, talk about mbed, talk about UF2

% % * programming languages for microcontrollers
% %     - discuss low level languages
% %     - talk about approaches to higher level languages by communities
% %     - talk about approaches to higher level languages by research
% %     - contextualise against our approach
% \subsection{Programming languages for novice users}


% Java-through-C compilation: \cite{varma2004java}

% VMSTAR: synthesizing scalable runtime environments for sensor networks \cite{koshy2005vmstar}

% % * Embedded Runtime Environments
% %     -largely remains as is.
% \subsection{Embedded Runtime Environments}

% Typically written in C and/or C++, environments for MCU programming all share a common design goal: to support developers by providing primitives and programming abstractions. Platforms can range from simple C++ classes that control hardware, to real-time operating systems (RTOSs) with scheduling and memory management.

% Arduino~\cite{buildingArduino2014} is an example of a simple platform where the developer uses high-level APIs to control hardware; there is no scheduler, and memory management is discouraged through a heavy emphasis on global variables.

% TinyOS~\cite{levis2005tinyos}, Contiki~\cite{dunkels2012contiki}, RIOT OS~\cite{baccelli2013riot}, Mynewt~\cite{ApacheMy53:online} mbed OS~\cite{ARMmbed}, Zephyr~\cite{HomeZeph63:online} are RTOS solutions known widely in the systems community. The majority focus on the networking features of sensor based devices and commonly adopt a preemptive scheduling model, which leads to competition over resources resolved using locks and condition synchronization primitives. Contiki has a cooperative scheduler, but uses proto-threads to store thread context --- local variables are not allowed as the context of the stack is not stored.

% Although the platforms above are widely used by C/C++ developers, none of these existing solutions align well with the programming paradigms seen in higher level languages. \CO bridges that semantic gap between the higher level language and the microcontroller, offering appropriate abstractions and higher level primitives written natively in C++.


% \subsection{Programming microcontrollers}

% \paragraph{Tethering a device} Scratch's device extensions~\cite{ScratchCACM2009} and Arduino's Firmata library~\cite{Firmata} take the tethered approach. However, in the educational and maker~\cite{dougherty2012maker} settings, MCUs often are embedded in projects. Here, this method has limited utility, as in such projects the MCU is powered by battery and is not connected to a host.

% \paragraph{Loading a pre-compiled native binary onto a device}
% Arduino~\cite{buildingArduino2014}, mbed~\cite{ARMmbed}, and other C-based solutions require installation of additional software and knowledge of low-level languages. These requirements do not suit a diverse demographic of inexperienced users.

% \paragraph{Interpretting pre-compiled bytecode on the device}
% Java for embedded systems~\cite{ClausenTOPLAS}

% suffer in terms of performance on small MCUs, operating slower than native C++ code, and consuming most of the resources (flash and RAM) of the MCU.

% \paragraph{Loading a compiler and virtual machine onto a device}
% MicroPython~\cite{MicroPython} and Espruino~\cite{espruinoBook};

% Our platform slots into options 2 and 3, using the \MC architecture and USB pen drives to avoid the need for installation of native applications or device drivers. Our use of STS provides better performance than the MicroPython and Espruino solution, as demonstrated in the evaluation,
% and allows us to fit in the very constrained space of the Uno.




% \subsection{Programming microcontrollers}

% talk about old protocols for flashing protocols, talk about mbed, talk about UF2
